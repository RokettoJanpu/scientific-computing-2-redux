\documentclass{article}
\usepackage{amsfonts, amsthm, amsmath, amssymb, mathtools, ulem, mathrsfs, physics, esint, siunitx, tikz-cd}
\usepackage{pdfpages, fullpage, color, microtype, cancel, textcomp, markdown, hyperref, graphicx}
\usepackage{enumitem}
\usepackage{algorithm}
\usepackage{algpseudocode}
\graphicspath{{./images/}}
\usepackage[english]{babel}
\usepackage[autostyle, english=american]{csquotes}
\MakeOuterQuote{"}
\usepackage{xparse}
\usepackage{tikz}

\usepackage{calligra}
\DeclareMathAlphabet{\mathcalligra}{T1}{calligra}{m}{n}
\DeclareFontShape{T1}{calligra}{m}{n}{<->s*[2.2]callig15}{}
\newcommand{\script}[1]{\ensuremath{\mathcalligra{#1}}}
\newcommand{\scr}{\script r}

% fonts
\def\mbb#1{\mathbb{#1}}
\def\mfk#1{\mathfrak{#1}}
\def\mbf#1{\mathbf{#1}}
\def\tbf#1{\textbf{#1}}

% common bold letters
\def\bP{\mbb{P}}
\def\bC{\mbb{C}}
\def\bH{\mbb{H}}
\def\bI{\mbb{I}}
\def\bR{\mbb{R}}
\def\bQ{\mbb{Q}}
\def\bZ{\mbb{Z}}
\def\bN{\mbb{N}}

% brackets
\newcommand{\br}[1]{\left(#1\right)}
\newcommand{\sbr}[1]{\left[#1\right]}
\newcommand{\brc}[1]{\left\{#1\right\}}
\newcommand{\lbr}[1]{\left\langle#1\right\rangle}

% vectors
\renewcommand{\i}{\hat{\imath}}
\renewcommand{\j}{\hat{\jmath}}
\renewcommand{\k}{\hat{k}}
\newcommand{\proj}[2]{\text{proj}_{#2}\br{#1}}
\newcommand{\m}[2][b]{\begin{#1matrix}#2\end{#1matrix}}
\newcommand{\arr}[3][\sbr]{#1{\begin{array}{#2}#3\end{array}}}

% misc
\NewDocumentCommand{\seq}{O{n} O{1} O{\infty} m}{\br{#4}_{{#1}={#2}}^{#3}}
\NewDocumentCommand{\app}{O{x} O{\infty}}{\xrightarrow{#1\to#2}}
\newcommand{\sm}{\setminus}
\newcommand{\sse}{\subseteq}
\renewcommand{\ss}{\subset}
\newcommand{\vn}{\varnothing}
\newcommand{\lc}{\epsilon_{ijk}}
\newcommand{\ep}{\epsilon}
\newcommand{\vp}{\varphi}
\renewcommand{\th}{\theta}
\newcommand{\cjg}[1]{\overline{#1}}
\newcommand{\inv}{^{-1}}
\DeclareMathOperator{\im}{im}
\DeclareMathOperator{\id}{id}
\newcommand{\ans}{\tbf{Ans. }}
\newcommand{\pf}{\tbf{Pf. }}
\newcommand{\imp}{\implies}
\newcommand{\impleft}{\reflectbox{$\implies$}}
\newcommand{\ck}{\frac1{4\pi\ep_0}}
\newcommand{\ckb}{4\pi\ep_0}
\newcommand{\sto}{\longrightarrow}
\DeclareMathOperator{\cl}{cl}
\DeclareMathOperator{\intt}{int}
\DeclareMathOperator{\bd}{bd}
\DeclareMathOperator{\Span}{span}
\newcommand{\floor}[1]{\left\lfloor#1\right\rfloor}
\newcommand{\ceil}[1]{\left\lceil#1\right\rceil}
\newcommand{\fxn}[5]{#1:\begin{array}{rcl}#2&\longrightarrow & #3\\[-0.5mm]#4&\longmapsto &#5\end{array}}
\newcommand{\sep}[1][.5cm]{\vspace{#1}}
\DeclareMathOperator{\card}{card}
\renewcommand{\ip}[2]{\lbr{#1,#2}}
\renewcommand{\bar}{\overline}
\DeclareMathOperator{\cis}{cis}
\DeclareMathOperator{\Arg}{Arg}
\newcommand{\om}{\omega}
\DeclareMathOperator{\diag}{diag}

% title
\title{Scientific Computing HW 4}
\author{Ryan Chen}
%\date{\today}
\setlength{\parindent}{0pt}


\begin{document}
	
\maketitle



\begin{enumerate}
	
	
	
\item The characteristic polynomial is
$$\rho(z) = z^k + \sum_{j=0}^{k-1}\alpha_jz^{k-1-j}$$
and the recurrence is
$$u_{n+1} + \sum_{j=0}^{k-1}\alpha_ju_{n-j} = 0$$
Define a linear operator $D:=r\dv{r}$. Then for all $n$ and $l$,
$$D[n^lr^n] = r\dv{r} n^lr^n = rn^lnr^{n-1} = n^{l+1}r^n$$
so that in particular,
$$n^lr^n = D[n^{l-1}r^n] = D^2[n^{l-2}r^n] = \ldots = D^l[r^n] \quad (1.1)$$
Plug $u_n=n^lr^n$ into the LHS of the recurrence, apply (1.1) to each term, then use the linearity of $D$.
$$u_{n+1} + \sum_{j=0}^{k-1}\alpha_ju_{n-j} = (n+1)^lr^{n+1} + \sum_{j=0}^{k-1}\alpha_j(n-j)^lr^{n-j}
= D^l[r^{n+1}] + \sum_{j=0}^{k-1}\alpha_j D^l[r^{n-j}]
= D^l\sbr{r^{n+1} + \sum_{j=0}^{k-1}\alpha_j r^{n-j}}$$
$$= D^l\sbr{r^{n-k+1}\br{r^k + \sum_{j=0}^{k-1}\alpha_jr^{k-1-j}}}
= D^l[r^{n-k+1}\rho(r)]
= D^l[0]
= 0$$



\item We first establish a lemma. For the $(k-1)\times k$ matrix
$$\m{\lambda & -1 \\ & \lambda & -1 \\ & & \ddots & \ddots \\ & & & \lambda & -1}$$
The submatrix formed by deleting the $j$th column has determinant $(-1)^{k-j}\lambda^{j-1}$. To see this, write the submatrix in block form.
$$M :=
\m{\m[]{\lambda & -1 \\ & \lambda & -1 \\ & & \ddots & \ddots \\ & & & \lambda & -1 \\ & & & & \lambda}
	& 0_{(j-1)\times(k-j)} \\
	0_{(k-j)\times(j-1)}
	& \m[]{-1 \\ \lambda & -1 \\ & \ddots & \ddots \\ & & \lambda & -1}}$$
Using the fact that
$$\det \m{B & 0 \\ 0 & C} = \det(B)\det(C)$$
we have $\det M = \lambda^{j-1}(-1)^{k-j}$.\\

We now aim to show $\det(\lambda I-A)=\rho(\lambda)$.
$$\lambda I-A = \m{
	\lambda & -1\\
	& \lambda & -1\\
	& & \ddots & \ddots\\
	& & & \lambda & -1\\
	\alpha_{k-1} & \alpha_{k-2} & \dots & \alpha_1 & \lambda+\alpha_0
}$$
Expanding over row $k$, starting at column $k$, and using the above lemma,
$$\det(\lambda I-A)
= (\lambda+\alpha_0)(-1)^{k-k}\lambda^{k-1} + \sum_{j=1}^{k-1}(-1)^{k+k-j}\alpha_j(-1)^{k-k+j}\lambda^{k-j-1}
= \lambda^k + \alpha_0\lambda^{k-1} + \sum_{j=1}^{k-1}(-1)^{2k}\alpha_j\lambda^{k-j-1}$$
$$= \lambda^k + \alpha_0\lambda^{k-1} + \sum_{j=1}^{k-1}\alpha_j\lambda^{k-j-1}
= \lambda^k + \sum_{j=0}^{k-1}\alpha_j\lambda^{k-j-1}
= \rho(\lambda)$$



\item Note: For explicit matrix norm computations we will use the Euclidean norm
$$\norm{A} = \sbr{\sum_{i,j}A_{ij}^2}^{1/2}$$

\pf Write the Jordan form of $A$.
$$A = SJS\inv,
\quad J = \diag(J_1,J_2,\dots,J_s),
\quad J_q = r_qI_{m_q} + N_q,
\quad N_q \overset{m_q\times m_q}{=} \m{
	0 & 1 & & \\
	& 0 & 1 & & \\
	& & \ddots & \ddots \\
	& & & 0 & 1
}$$
For all $p$,
$$A^p = SJ^pS\inv,
\quad J^p = \diag(J_1^p,J_2^p,\dots,J_s^p)$$
so we seek a bound on $\norm{J^p}$, where $J=rI+N$ is a Jordan block for an eigenvalue $r$ with multiplicity $m$ (we temporarily use $J$ to denote a single block instead of the whole matrix).

If $m=1$ then $J=[r]$, and since $\rho$ satisfies the root condition, $|r|\le 1$, giving
$$\norm{J^p} = |r|^p \le 1$$
Now say $m>1$, so that $|r|<1$ by the root condition. If $r=0$ then $J=N$, hence $\norm{J^p}=0$ for all $p\ge m$. Now say $r\ne 0$. For $p\ge m$,
$$J^p = (rI+N)^p = \sum_{j=0}^p \binom{p}{j}r^{p-j}N^j = \sum_{j=0}^m \binom{p}{j}r^{p-j}N^j$$
$$\imp \norm{J^p} \le \sum_{j=0}^p \binom{p}{j}|r|^{p-j}\norm{N^j}$$
Fix $j$ with $0\le j\le m$. We establish a bound on $\binom{p}{j}|r|^{p-j}$. Note
$$\binom{p}{j}|r|^{p-j} = \frac{p!}{j!(p-j)!}|r|^p|r|^{-j}
= \frac{1}{|r|^jj!}|r|^p \prod_{i=0}^{j-1}(p-i)
= \frac{1}{|r|^jj!} \cdot \frac{\prod_{i=0}^{j-1}(p-i)}{|1/r|^p}$$
Considering the fraction in the RHS as $p\to\infty$, the denominator, an exponential in $p$ with base $|1/r|>1$, grows faster than the numerator, a polynomial in $p$. This means $\binom{p}{j}|r|^{p-j}$ tends to 0 as $p\to\infty$, hence it has a bound $M_j$ independent of $p$.

Writing small powers of $N$,
$$N^0 = I = \m{
	1 &  & & \\
	& 1 &  & & \\
	& & \ddots &  \\
	& & & 1 & 
},
\quad N = \m{
	0 & 1 & & \\
	& 0 & 1 & & \\
	& & \ddots & \ddots \\
	& & & 0 & 1
},
\quad N^2 = \m{
	0 & 0 & 1 & \\
	& 0 & 0 & 1 & \\
	& & 0 & \ddots & 1 \\
	& & & 0 & 0
},
\quad N^3 = \m{
	0 & 0 & 0 & 1 \\
	& 0 & 0 & 0 & 1 \\
	& & 0 & 0 & 0 \\
	& & & 0 & 0
}
$$
we see that for $0\le j\le m$,
$$\norm{N^j} = \sbr{\sum_{i,j}(N^j)_{ij}^2}^{1/2} = [m-j]^{1/2} \le m^{1/2}$$

We have a bound on $\norm{J^p}$ independent of $p$. Here we re--introduce the subscript $q$.
$$\norm{J_q^p} \le \max\sbr{\norm{J_q^1},\norm{J_q^2},\dots,\norm{J_q^{m_q-1}},\sum_{j=0}^{m_q} M_{q,j} m_q^{1/2}} =: K_q$$
Now looking at the whole Jordan matrix $J$,
$$\norm{J^p} \le \sum_{q=1}^s \norm{J_q^p} \le \sum_{q=1}^s K_q$$
Thus
$$\norm{A^p} \le \norm{S}\norm{S\inv} \sum_{q=1}^s K_q$$



\item Plug the supposed solution $U_n$ into the LHS of the recurrence.
$$U_{n+1} = A^{n-k+2}U_{k-1} + h\sum_{j=0}^{n+1-k}A^jG_{n-j}
= A^{n-k+2}U_{k-1} + hG_n + h\sum_{j=1}^{n+1-k}A^jG_{n-j}$$
$$= A^{n-k+2}U_{k-1} + hG_n + h\sum_{j=0}^{n-k}A^{j+1}G_{n-j-1}$$
Plug the solution into the RHS.
$$AU_n + hG_n = A^{n-k+2}U_{k-1} + h\sum_{j=0}^{n-k}A^{j+1}G_{n-j-1} + hG_n$$
The two expressions are equal, so $U_n$ indeed solves the recurrence.



\item

\begin{enumerate}[label=(\alph*)]
	
	\item For BDF2, the interpolant is
	$$p(t) = y_{n+1} + y[t_{n+1},t_n](t-t_{n+1}) + y[t_{n+1},t_n,t_{n-1}](t-t_{n+1})(t-t_n)$$
	The derivative of the last term at $t_{n+1}$ is
	$$\dv{t}\eval_{t=t_{n+1}} (t-t_{n+1})(t-t_n)
	= (t-t_{n})+(t-t_{n+1})\eval_{t=t_{n+1}}
	= 2t_{n+1}-t_{n}-t_{n+1}
	= t_{n+1}-t_n
	= h_n$$
	and so
	$$p'(t_{n+1}) = y[t_{n+1},t_n] + y[t_{n+1},t_n,t_{n-1}]h_n$$
	Equating this with $f_{n+1}:=f(t_{n+1},u_{n+1})$ gives the BDF2 method.
	$$\frac{u_{n+1}-u_n}{h_n} + \frac{\frac{u_{n+1}-u_n}{h_n} - \frac{u_n-u_{n-1}}{h_{n-1}}}{h_n+h_{n-1}}h_n = f_{n+1}$$
	From $h_n=t_{n+1}-t_n$ and $\om=\frac{h_n}{h_{n-1}}$,
	$$1 + \om = 1 + \frac{h_n}{h_{n-1}} = \frac{h_{n-1}+h_n}{h_{n-1}}
	\imp (1+\om)^2 = \frac{1}{h_{n-1}^2}\sbr{h_{n-1}^2+h_n^2+2h_{n-1}h_n} \quad (5.1)$$
	$$1 + 2\om = 1 + 2\frac{h_n}{h_{n-1}} = \frac{h_{n-1}+2h_n}{h_{n-1}} \quad (5.2)$$
	From (5.1) and (5.2),
	$$\frac{(1+\om)^2}{1+2\om} = \frac{h_{n-1}}{h_{n-1}+2h_n}\frac{1}{h_{n-1}^2}\sbr{h_{n-1}^2+h_n^2+2h_{n-1}h_n}
	= \frac{1}{h_{n-1}+2h_n}\sbr{h_{n-1}+\frac{h_n^2}{h_{n-1}}+2h_n}$$
	$$= h_{n-1}(1+2\om)\sbr{h_{n-1}+\frac{h_n^2}{h_{n-1}}+2h_n} \quad (5.3)$$
	Now multiply the method by $h_n(h_n+h_{n-1})$.
	$$(u_{n+1}-u_n)(h_n+h_{n-1}) + \sbr{\frac{u_{n+1}-u_n}{h_n} - \frac{u_n-u_{n-1}}{h_{n-1}}}h_n^2 = h_n(h_n+h_{n-1})f_{n+1}$$
	Collect coefficients of the following terms. Rewrite them using (5.2) and (5.3).
	\begin{align*}
		u_{n+1} &: \quad h_n + h_{n-1} + \frac{h_n^2}{h_n} = 2h_n + h_{n-1} \overset{(5.2)}{=} h_{n-1}(1+2\om) \\
		u_n &: \quad -(h_n + h_{n-1}) + \sbr{-\frac{1}{h_n} - \frac{1}{h_{n-1}}}h_n^2 = -\sbr{2h_n + h_{n-1} + \frac{h_n^2}{h_{n-1}}} \overset{(5.3)}{=} -h_{n-1}(1+2\om)\frac{(1+\om)^2}{1+2\om} \\
		u_{n-1} &: \quad \frac{h_n^2}{h_{n-1}} = h_{n-1}\om^2
	\end{align*}
	Putting these together, the method is
	$$h_{n-1}(1+2\om)u_{n+1} - h_{n-1}(1+2\om)\frac{(1+\om)^2}{1+2\om}u_n + h_{n-1}\om^2 = h_n(h_n+h_{n-1})f_{n+1}$$
	Divide by $h_{n-1}(1+2\om)$.
	$$u_{n+1} - \frac{(1+\om)^2}{1+2\om}u_n + \frac{\om^2}{1+2\om}u_{n-1} = h_n\frac{\frac{h_n}{h_{n-1}}+1}{1+2\om}f_{n+1}
	= h_n\frac{1+\om}{1+2\om}f_{n+1}$$
	
	
	\item From the LHS of the method, define
	$$\rho(z) := z^2 - \frac{(1+\om)^2}{1+2\om}z + \frac{\om^2}{1+2\om}$$
	By Vieta's formulas, the roots $r,s$ of $\rho$ satisfy
	$$r + s = \frac{(1+\om)^2}{1+2\om} = \frac{1+\om^2+2\om}{1+2\om} = \frac{\om^2}{1+2\om} + 1$$
	$$rs = \frac{\om^2}{1+2\om}$$
	From these we see
	$$r = \frac{\om^2}{1+2\om},
	\quad s = 1$$
	Since $\om<1+\sqrt2$,
	$$w^2 < (1+\sqrt2)^2 = 1 + 2 + 2\cdot\sqrt2 = 1 + 2(1+\sqrt2) = 1 + 2\om
	\imp 0 \le r = \frac{\om^2}{1+2\om} < 1
	\imp |r|<1$$
	This leaves $s=1$ as the only root with modulus 1, and it has multiplicity 1. Thus $\rho$ satisfies the root condition, hence the method is stable.
	
\end{enumerate}	
	
\end{enumerate}

\end{document}